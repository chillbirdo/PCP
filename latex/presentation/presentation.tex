\documentclass{beamer}
\usepackage{beamerthemesplit} % new 
\usepackage[utf8x]{inputenc}
\usepackage[ngerman]{babel} 
\begin{document}
\title{A Hybrid Algorithm for the Partition Coloring Problem} 
\author{Gilbert Fritz} 
\date{\today} 

\frame{\titlepage} 

\frame{\frametitle{Table of contents}\tableofcontents} 


\section{Introduction} 
\frame{\frametitle{Introduction} 
\begin{itemize}
\item was soll ich hier eigentlich sagen?  
\item soll ich ein TU Wien template verwenden?
\end{itemize} 
}


\section{Problem Description} 
\subsection{The Partition Coloring Problem}
\frame{\frametitle{Partition Coloring Problem}
\begin{itemize}
\item Problem erklären  
\item anhand von Grafiken
\end{itemize} 
}

\subsection{Routing and Wavelength Assignment}
\frame{\frametitle{Routing and Wavelength Assignment}
\begin{enumerate}
\item Hintergrund erklären.
\item Bezug zu PCP erklären.
\item Kurz min-RWA erklären. 
\end{enumerate}
}


\section{Previous work} 
\frame{\frametitle{Routing and Wavelength Assignment}
\begin{enumerate}
\item Alle Papers kurz kommentieren.
\item TabuSearch von Nohora genauer zeigen und erklären.
\end{enumerate}
}

\section{My Approach}
\subsection{Idea}
\frame{\frametitle{Idea}
\begin{enumerate}
\item meine Idee vortragen
\item und grafisch erläutern
\end{enumerate}
}
\subsection{Algorithms}
\frame{\frametitle{Alg1: GesamtAlgorithmus}
\begin{enumerate}
\item Pseudocode zeigen und erklären
\end{enumerate}
}
\frame{\frametitle{Alg2: Lokale Suche}
\begin{enumerate}
\item Pseudocode zeigen und erklären
\end{enumerate}
}
\frame{\frametitle{Alg3: Recoloring with OneStepCD}
\begin{enumerate}
\item OneStepCD Original und Abänderung zeigen
\end{enumerate}
}
\frame{\frametitle{ILP1: Recoloring with ILP1}
\begin{enumerate}
\item ILP1 zeigen und erklären
\end{enumerate}
}
\frame{\frametitle{ILP2: Recoloring with ILP2}
\begin{enumerate}
\item ILP2 zeigen und erklären
\end{enumerate}
}

\section{Results}
\frame{\frametitle{Results}
\begin{enumerate}
\item Ergebnisse von Random, OneStepCD, ILP1, ILP2
\item Beim Zufallsbasierten worst, best, avg
\end{enumerate}
\begin{tabular}{|c|c|c|}
\hline
\textbf{Instance} & \textbf{Runtime} & \textbf{Result} \\
\hline
a.pcp & 0.01 & 3  \\
\hline
b.pcp & 20.34 & 9 \\
\hline
\end{tabular}}

\section{Conclusion}
\frame{\frametitle{Conclusion}
\begin{enumerate}
\item aufwand in die wiedereinfärbung zu stecken hat nichts genutzt.
\end{enumerate}
}


\end{document}
