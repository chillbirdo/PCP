\documentclass{beamer}
\usepackage{beamerthemesplit} % new 
\usepackage[utf8x]{inputenc}
\usepackage[ngerman]{babel} 

\usepackage{epsfig}
\usepackage{amsmath}
\usepackage{placeins}
\usepackage{float}

\usepackage{algorithmic}
\usepackage[linesnumbered,ruled,vlined]{algorithm2e}


\begin{document}
\title{A Hybrid Algorithm for the Partition Coloring Problem} 
\author{Gilbert Fritz} 
\date{\today} 

\frame{\titlepage} 

\frame{\frametitle{Table of contents}\tableofcontents} 


\section{Problem Beschreibung} 
\subsection{Partition Coloring Problem}
\frame{\frametitle{Partition Coloring Problem}
\begin{itemize}
\item Problem erklären  
\item anhand von Grafiken
\end{itemize} 
}

\subsection{Routing and Wavelength Assignment}
\frame{\frametitle{Routing and Wavelength Assignment}
\begin{enumerate}
\item Hintergrund erklären.
\item Bezug zu PCP erklären.
\item Kurz min-RWA erklären. 
\end{enumerate}
}


\section{Bestehende Arbeiten} 
\subsection{Exakte Verfahren}
\frame{\frametitle{Exakte Verfahren}
\begin{enumerate}
\item Frota, Ribiero (2010): Branch-And-Cut Algorithmus basierend auf der Formulierung durch Repräsentative
\end{enumerate}
}
\subsection{Heuristische Verfahren}
\frame{\frametitle{Heuristische Verfahren}
\begin{enumerate}
\item Li, Shima (2000): Einige Greedy Heuristiken, u.a. OneStepCD
\item Noronha, Ribiero (2006): Tabusuche
\item Hoshino, Souza, Frota (2011): Branch-And-Price
\item Hu, Raidl, Pop (2013): Memetischer Algorithmus
\end{enumerate}
}

\section{Eigener Ansatz}
\subsection{Übersicht}
\frame{\frametitle{Gesamtalgorithmus: Vorgehen}
\begin{enumerate}
\item Startheuristik: OneStepCD
\item Für jede Menge $U$ an Knoten gleicher Farbe $c$: Entfärbe und färbe ohne $c$ neu ein mittels $\{RANDOM, OneStepCD, ILP1, ILP2\}$
\item Die Lösungen pro Farbe nach Anzahl der Konflikte sortieren.
\item Für jede Lösung versuche Konflikte mittels Tabusuche zu eliminieren. Alle Knoten-Farb-Paare kommen von Beginn an für eine gewisse Anzahl an Iterationen auf die Tabuliste.
\item Konnten bei einer Lösung alle Konflikte elminiert werden, beginne wieder bei 2, sonst gib die letzte zulässige Lösung zurück.
\end{enumerate}
}

\begin{frame}[fragile]
\frametitle{Pseudocode Gesamtalgorithmus}
\includegraphics[scale=0.32]{gesamtalgo.png}
\end{frame}

\subsection{Startlösung}
\begin{frame}
\frametitle{Startlösung: OneStepCD}
  \begin{algorithm}[H]
  \scriptsize
  \KwIn{An uncolored Graph $G=(V,E)$}
  \KwOut{A feasible Coloring $V'$}
  Remove from G all edges $(i,j) \in E$ : $i,j \in V_k$ for some $k=1,\ldots,q$\; 
  Set $V' \gets \emptyset $\;
  \While{$|V'| < q$} {
    Set $X \gets \emptyset $\;
    \For{$k=1,\ldots,q$ : $V_k \cap V'=0$}{
      Set $X \gets X \cup argmin\{CD(i) : i \in V_k\}$\; 
    }
    Set $x \gets argmax\{CD(i) : i \in X \}$\;
    Set $V' \gets V' \cup \{x\}$\;
    Assign the minimum possible colour to x\;
    Remove from G all nodes in $V_{c(x)} \setminus \{x\} $\;
  }
  \Return{$V'$}\;
  \caption{{\sc OneStepCD}}
  \label{algo:osdc}
  \end{algorithm}
\end{frame}

\FloatBarrier
\subsection{Neueinfärbung}
\begin{frame}
\frametitle{Random Recoloring}
\begin{enumerate}
\item Knoten werden zufällig neu eingefärbt.
\end{enumerate}

\end{frame}

\frame{\frametitle{Recoloring with OneStepCD}
\begin{algorithm}[H]
\scriptsize
\KwIn{A partial Solution $P$, a number of maximum colours $cmax$ }
\KwOut{A feasible Solution $S$}
Let $U$ be the set of uncolored nodes in $P$\;
Set $S \gets \emptyset$\;
\While{$|U| > 0$} {
  Set $X \gets \emptyset $\;
  \For{ $u \in U$}{
    $X \gets X \cup argmin\{CD(i) : i \in V_{c(u)}\}$\; 
  }
  Set $x \gets argmax\{CD(i) : i \in X \}$\;
  Set $cmin \gets$ the minimum possible colour that can be assigned to x\;
  \If{$cmin \geq cmax$}{
    $cmin \gets$ the color that produces the fewest conflicts.
  }
  Assign $cmin$ to $x$\;
  $S \gets S \cup \{x\}$\;
  $U \gets U \setminus V_{c(x)}$\;
}
\Return{$V'$}\;
\caption{{\sc OneStepCD Recoloring}}
\label{algo:osdc2}
\end{algorithm}

}
\frame{\frametitle{Recoloring with ILP1}

Let $Q = {Q_1,\ldots,Q_q}$ be the set of Clusters. Every cluster $Q_p$ consists of a set of nodes. Let $C=\{1,\ldots,cmax\}$ be the
set of allowed colors. Let $M$ be a 3-dimensional array of constants, storing
for every cluster $p \in Q$, the number conflicts that would occur by selecting the pair $(v \in Q_p, c \in C)$.
$E$ denotes the set of edges and $P[v]$ the cluster of node $v$.


\begin{equation*}
\scriptsize
\begin{aligned}
& \underset{X}{\text{minimize}} && \sum_{p \in Q}\sum_{v \in Q_p}\sum_{c \in C} X_{pvc} * M_{pvc}                    &&&(1)\\
& \text{subject to} && \sum_{v \in Q_p}\sum_{c \in C} X_{pvc}=1, && \forall p \in Q    &(2)\\
&&& X_{pvc}+X_{quc} \leq 1, && \forall ((p,v),(q,u)) \in E, \forall c \in C     &(3)\\
&&& X_{pvc} \in \{0,1\}, && \forall p \in Q, \forall v \in Q_p, \forall c \in C         &(4)
\end{aligned}
\end{equation*}
}


\frame{\frametitle{Recoloring with ILP2}

Let $U$ be the set of uncolored nodes in uncolored clusters and $color[(p,v)]$ the color of the node $v$ in partition $p$.

\begin{equation*}
\scriptsize
\begin{aligned}
& \underset{Z}{\text{minimize}} && \sum_{p \in Q}\sum_{v \in Q_p}\sum_{c \in C} Z_{pvc}                                              &&&(1)\\
& \text{subject to} && Z_{pvc} \geq X_{quc}, && \forall ((p,v),(q,u))\in E : (p,v) \notin U, (q,u) \in U, c=color[(p,v)]                                                            &(2)\\
&&& \sum_{v \in Q_p}\sum_{c \in C} X_{pvc}=1, && \forall p \in Q   &(3)\\
&&& X_{pvc}+X_{quc} \leq 1, && \forall ((p,v),(q,u)) \in E, \forall c \in C     &(4)\\
&&& X_{pvc} \in \{0,1\}, && \forall p \in Q, \forall v \in Q_p, \forall c \in C        &(5)
\end{aligned}
\end{equation*}

}

\subsection{Konflikte eliminieren: Tabusuche}
\begin{frame}[fragile]
\frametitle{Pseudocode Tabusuche}
\includegraphics[scale=0.32]{tabusearch.png}

\end{frame}


\section{Resultate}
\frame{\frametitle{Results}
\begin{enumerate}
\item Ergebnisse von Random, OneStepCD, ILP1, ILP2
\item Beim Zufallsbasierten worst, best, avg
\end{enumerate}
\begin{tabular}{|c|c|c|}
\hline
\textbf{Instance} & \textbf{Runtime} & \textbf{Result} \\
\hline
a.pcp & 0.01 & 3  \\
\hline
b.pcp & 20.34 & 9 \\
\hline
\end{tabular}}

\section{Zusammenfassung}
\frame{\frametitle{Zusammenfassung}
\begin{enumerate}
\item Beim Reduzieren der Farben spiel die Tabusuche eine weit wichtigere Rolle als die Neueinfärbung.
\end{enumerate}
}


\end{document}
