\documentclass{article}

\usepackage[utf8x]{inputenc}
\usepackage[ngerman]{babel} 

\usepackage{epsfig}
\usepackage{amsmath}
\usepackage{placeins}
\usepackage{float}

\usepackage{algorithmic}
\usepackage[linesnumbered,ruled,vlined]{algorithm2e}


\title{Ein Hybrider Algorithmus für das Partition Coloring Problem}
\author{Gilbert Fritz, 0827276 / 066931}


\begin{document}

\section{Tables}

 \begin{table}
      \begin{tiny}
      \textit{500-1.in, mit verschiedenen Tabusizes} 
      \end{tiny}
      
      \resizebox{\columnwidth}{!}{%
      \begin{tabular}{|l|l||l|l|l|}\hline
      \multicolumn{2}{|l||}{Instance set}&\multicolumn{3}{|l|}{Random}\\
      \cline{1-5}
      tsMin & tsMax & $\overline{obj}$ & $sd$ & $\overline{time}$\\
      \hline
            
      0.25 & 0.75 & \textbf{53.0} & 0.000 & 19.535 \\
      0.0 & 1.0 & \textbf{52.8} & 0.000 & 21.058 \\
      0.0 & 0.5 & \textbf{53.0} & 0.000 & 18.278 \\
      0.5 & 1.0 & \textbf{52.9} & 0.000 & 20.997 \\
      0.25 & 1.0 & \textbf{53.1} & 0.000 & 19.455 \\
      0.0 & 0.75 & \textbf{53.0} & 0.000 & 20.211 \\
      \hline
      \end{tabular}
      }
      \end{table}

\end{document}
