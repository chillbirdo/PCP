\documentstyle{article}


\title{Ein Hybrider Algorithmus für das Partition Coloring Problem}
\author{Gilbert Fritz, 0827276 / 066931}


\begin{document}
\maketitle
\nocite{li-00}

\section{Problemstellung}

Diese Arbeit besch\"aftigt sicht mit dem Partition Coloring Problem (PCP). Es handelt sich dabei um eine Generalisierung des Knotenf\"arbungsproblems und ist ein Optimierungsproblem der Komplexit\"atsklasse NP.\\
Gegeben ist ein Graph, dessen Knotenmenge in disjunkte Partitionen unterteilt ist. Aus jeder Partition muss ein Knoten gew\"ahlt werden. Der aus den gew\"ahlten Knoten entstandene Subgraph soll unter der Bedingung eingef\"arbt werden, dass kein zueinander adjazentes Knotenpaar die gleiche Farbe annimmt.\\
Zur L\"osung dieses Problems sollen mittels heuristischer Verfahren initiale L\"osungen erstellt und diese mittels Neueinf\"arbung und Lokaler Suche verbessert werden. Das Problem der Neueinf\"arbung soll sowohl heuristisch, als auch exakt gel\"ost werden.

\section{Erwartetes Resultat}

Mittels des beschriebenen, hybriden Verfahrens sollen in annehmbarer Zeit L\"osungen m\"oglichst nahe am Optimum gefunden werden. Weiters soll \"uberpr\"uft werden, ob ein exakter L\"osungsansatz mittels mathematischer Programmierung beim Teilproblems der Neueinf\"arbung zu Verbesserungen führt. Das Ziel der Arbeit ist es, einen alternativen L\"sungsansatz zu den bereits Bestehenden zu erforschen.


\section{Methodisches Vorgehen}




\section{State-of-the-art}

blabla

\section{Bezug zum oben angeführten Studium}

Bei meinem Studium "Computational Intelligence" habe ich mich vorwiegend auf den Bereich Algorithmik konzentriert - in jenen Bereich f\"allt auch der Inhalt dieser Arbeit. Aufbauend auf den Lehrveranstaltungen \textit{Algorithmen und Datenstrukturen 1 und 2}, verschafften mir die Lehrveranstaltungen \textit{Algorithmen auf Graphen}, \textit{Problem Solving and Search in Artificial Intelligence}, sowie \textit{Heuristische Optimierungsverfahren} die n\"otigen Voraussetzungen vielf\"altige L\"osungsans\"atze bei der Bearbeitung des "Partition Coloring Problems" in Betracht zu ziehen. F\"ur die Neueinf\"arbung von Teilgraphen wende ich unter anderem exakte Verfahren mittels mathematischer Programmierung an. Die Grundlagen dazu erlernte ich in den Lehrveranstaltungen \textit{Fortgeschrittene Algorithmen und Datenstrukturen} und \textit{Modeling and Solving Constrained Optimization Problems}. Die Besch\"aftigung mit Problemen, die weiters in \textit{Effiziente Algorithmen} sowie \textit{Approximationsalgorithmen} behandelt wurden, erg\"anzen meine F\"ahigkeit in diesem Gebiet.

\bibliographystyle{unsrt}   % this means that the order of references
			    % is dtermined by the order in which the
			    % \cite and \nocite commands appear
\bibliography{pcp}  % list here all the bibliographies that
			     % you need.
			     			      
\end{document}
