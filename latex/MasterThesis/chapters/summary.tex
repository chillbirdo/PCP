
The PCP is a quite recently proposed COP which generalizes the classical VCP by considering the possiblity to select subsets of nodes. While for the VCP much research has been done, only a few papers about the PCP has been published so far. In this work a strategy is presented that creates an initial solution by a heuristical algorithm and improves the solution quality by recoloring sets of nodes of same color before eliminating the resulting conflicts by applying a tabu search. It has been tried to enhance the algorithm presented in \cite{noronha-06} by substituting the process of random recoloring by more sophisticated algorithms in order to minimize the number of resulting conflicts. Therefor a variation of the $\mathit{OneStepCD}$ algorithm \cite{li-00} and two ILPs were used. A local search algorithm then tries to eliminate all these conflicting nodes to create a feasible solution. Furthermore experiments with variations of the ILPs and a mechanism that puts the most recently recolored subgraph on the tabulist for an amount of iteration in order to protect the coloring of that subgraph from being overwritten have been done.\\\\
The results have shown that more sophisticated recoloring algorithms can reduce the number of conflicts dramatically. For the instances used, a random recoloring produces an amount of conflicting nodes up to $7.5$ times higher than an optimized recoloring does. The fact that this gap is not reflected significantly in the final results leads to the conclusion that for the presented strategy the tabu search is much more relevant than the recoloring process. Finally an alternative strategy, that is suspected by the author to be more suitable for sophisticated recoloring methods has been proposed.
