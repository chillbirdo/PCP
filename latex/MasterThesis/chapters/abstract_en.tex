\chapter*{Abstract}

The Partition Coloring Problem (PCP) is a generalization of the the classical Vertex Coloring Problem (VCP), partitioning the set of nodes into clusters and seeking a coloring for the subgraph induced by selecting exactly one node of each cluster. It is a member of the class of so called $\mathcal{NP}$-hard problems, i.e. problems that are computationally complex. Therefore

One of the real world applications for PCP is the assignment of wavelengths to connections of all optical networks.


Diese Arbeit beschäftigt sich mit dem Partition Coloring Problem (PCP). Es handelt sich dabei um eine Generalisierung des Knotenfärbungsproblems und ist ein Optimierungsproblem der Komplexitätsklasse $\mathcal{NP}$.\\
Gegeben ist ein Graph, dessen Knotenmenge in disjunkte Partitionen unterteilt ist. Aus jeder Partition muss ein Knoten gewählt werden. Der durch die gewählten Knoten induzierte Subgraph soll unter der Bedingung eingefärbt werden, dass kein zueinander adjazentes Knotenpaar die gleiche Farbe annimmt. Ziel ist es, die Gesamtanzahl der verwendeten Farben - die sogenannte chromatische Zahl - zu minimieren.\\
Zur Lösung dieses Problems sollen mittels heuristischer Verfahren initiale Lösun\-gen erstellt und diese mittels Tabusuche und wiederholter, partieller Neueinfär\-bung verbessert werden. Das Problem der Neueinfärbung wird mit unterschiedlichen Ansätzen gelöst. 
