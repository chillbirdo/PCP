\chapter*{Abstract}

The Partition Coloring Problem (PCP) generalizes the classical Vertex Coloring Problem (VCP) by partitioning the set of nodes into clusters and aims to find a coloring for the subgraph induced by selecting exactly one node of each cluster. It is a member of the class of so called $\mathcal{NP}$-hard problems, i.e. problems that are computationally complex. One of the real world applications for PCP is the assignment of wavelengths to connections of all optical computer networks, as they appear as backbones of the internet infrastructure of our time.\\

This thesis presents an approach to tackle the PCP by means of metaheuristics. An improvement strategy is presented, where specific subgraphs are reselected and recolored, temporarly allowing infeasible solutions. Feasability is then acquired by using a tabu search. The main innovation of the approach is the effort that is put in the subproblem of reselection/reassignment of subgraphs, where a heuristic and two  Integer Linear Program (ILP) formulations are applied. The algorithm is evaluated using different parameter settings as well as slight variations, the results are compared to each other as well as to previous works. Further experiments have been made on gaining initial solutions, comparing two already known algorithms \textit{OneStepCD} and an adaptation of \textit{DANGER}. The algorithm competes with the best heuristics known so far in terms of solution quality and runtime. A reflection of the approach as well as a proposal for improvement is outlined.
