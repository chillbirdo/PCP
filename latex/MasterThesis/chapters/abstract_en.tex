\chapter*{Abstract}

The Partition Coloring Problem (PCP) generalizes the classical Vertex Coloring Problem (VCP) by partitioning the set of nodes into clusters and aims to find a coloring for the subgraph induced by selecting exactly one node of each cluster. It is a member of the class of so called $\mathcal{NP}$-hard problems, i.e. problems without a known algorithm for solving it efficiently. One of the real world applications for PCP is the assignment of wavelengths to data transmitting connections of all optical computer networks, as they appear as backbones of the Internet infrastructure of our time. In opposition to VCP, not much research has been investigated in PCP so far.\\

This thesis presents an approach \textit{Hybrid-PCP} tackling the PCP by means of metaheuristics combined with exact approaches for solving subproblems. Therefore an improvement strategy is applied, where nodes in specific subgraphs are reselected and recolored, temporarily allowing infeasible solutions. Feasibility is then reacquired by using a tabu search. The main innovation of the approach is the effort that is put in the process of node reselection and reassignment, where a heuristic and two  Integer Linear Program (ILP) formulations are used. The algorithm is evaluated using different parameter settings as well as slight variations, the results are compared to each other as well as to previous works. Further experiments have been made on gaining initial solutions, comparing two already known algorithms \textit{OneStepCD} and an adaptation of \textit{DANGER}. \textit{Hybrid-PCP} can compete with the best heuristics known so far in terms of solution quality and runtime. A reflection of the approach as well as a proposal for improvement is explained.

