\chapter*{Kurzfassung}

Das Partition Coloring Problem (PCP) generalisiert das Vertex Coloring Problem (VCP) durch Unterteilung der Knotenmenge in Gruppen und besteht aus der Berechnung einer Knotenfärbung des durch Selektion genau eines Knotens pro Gruppe induzierten Subgraphens. Das PCP gehört zur Gruppe der so genannten $\mathit{NP}$-schweren Probleme, i.e. Probleme für die kein effizientes Verfahren zur Berechnung einer exakten Lösung besteht. Eine seiner Anwendungen besteht in der Zuordnung von Wellenlängen zu Datenübertragungsleitungen optischer Computernetzwerke, wie sie heute beispielsweise als Backbone in der Infrastruktur des Internets vorkommen. Im Gegensatz zum VCP bleibt das PCP bis heute wenig erforscht.\\

Diese Arbeit präsentiert ein metaheuristisches Verfahren \textit{Hybrid-PCP} zur Lösung des PCP, dass sich zusätzlich exakter Methoden bedient um Teilprobleme zu lösen. Dabei wird eine Verbesserungsstrategie verfolgt, in der Knoten in spezifischen Teilgraphen neu selektiert und eingefärbt werden, wobei temporär auch ungültige Lösungen zugelassen werden. Die Gültigkeit wird danach mittels Tabusuche wiederhergestellt. Die Hauptinnovation dieser Arbeit liegt im Aufwand, der für die Neuselektion und -einfärbung der Teilgraphen aufgewendet wird, als dass dafür eine Heuristik und zwei mathematische Programmformulierungen eingesetzt werden. Der Algorithmus wird mit unterschiedlicher Parametrisierung als auch leichten Variationen evaluiert, die Ergebnisse miteinander und mit solcher vorhergehender Arbeiten verglichen. Weitere Experimente werden mit der Erstellung initialer Lösungen durchgeführt, wobei zwei bereits bekannte Algorithmen \textit{OneStepCD} und eine Adaption des für VCP entwickelten \textit{DANGER} Algorithmus miteinander verglichen werden. \textit{Hybrid-PCP} kann betreffend Lösungsqualität und Laufzeit mit den besten bisher gefundener Lösungsansätze konkurrieren. Der Autor legt weiters eine Reflexion des Verfahrens sowie einen Verbesserungsvorschlag dar.
