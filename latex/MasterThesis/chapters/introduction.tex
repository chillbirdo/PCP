\section{Motivation}

In order to obey the emerging demand for advanced broadband Internet applications such as video-conferences, high performance computing and others, extensive networks capacities have to be achieved. Links in optical networks operate much faster than their currently available electronic counterparts. Combined with the technique of Wavelength Division Multiplexing (WDM), which permits the simultaneous transmission of different channels along the same fiber \cite{noronha-06}, these so called Wavelength Routed Optical Networks (WRON's) are promising candidates for providing a flexible transport backbone network \cite{krishnaswamy-01}. They bring out new problems in coordination of wavelengths usage \cite{murthy-02}. One of them is the Routing and Wavelength Assignment Problem (RWA), which consists in routing a set of light-paths and assigning a wavelength to each of them. The variant where all connection requirements are known beforehands and which aims to minimize the amount of used wavelength is called min-RWA and found to be $\mathcal{NP}$-hard \cite{erlebach-01}.\\\\
Assigning wavelengths to one out of many paths for each connection requirement is equivalent to the $\mathcal{NP}$-hard \textbf{Partition Coloring Problem (PCP)} \cite{li-00} which is subject of this thesis. Given a graph consisting of a clustered set of vertices and a set of edges, the aim is to select one vertex per cluster and for each vertex in the resulting set assign a color in the way that the overall number of colors -- which in this context is said to be the chromatic number -- is minimized. If each cluster holds only one vertex, the problem reduces to the Standard Vertex Coloring Problem (VCP), which has been studied extensively and is used for a wide range of applications as scheduling, register allocation, pattern matching and others. In contrast, only a few papers have been published on PCP so far.\\\\
The aim of this thesis is to explore a solution method that has not been subject of publications on PCP yet,	in order to achieve better approximization methods for the PCP.
%Because of its complexity, the PCP counts to the class of $\mathcal{NP}$-optimization problems (NPO)


\section{Guide to the Thesis}

Necessary theoretical fundamentals like definitions, terms and methods are introduced in Chapter \ref{ch:prelim}. Afterwards, Chapter \ref{ch:problem} defines the PCP as well as the min-RWA formally and comments their computational complexity. Previous works and related research done so far is presented in Chapter \ref{ch:previous}. Chapter \ref{ch:approach} provides details of the approach developed for the PCP and Chapter \ref{ch:results} presents its experimental results. Chapter \ref{ch:summary} summarizes the knowledge achieved within this thesis, brings the considered approach into question and finally proposes a possible further work.