\section{Motivation}

\section{Problem Definition}

A master's thesis at the Faculty of Informatics has to be finished within six months. During this period regular meetings between the advisor(s) and the author have to take place.
In addition, the following milestones have to be fulfilled:
\begin{enumerate}
  \item  Within one month after having fixed the topic of the thesis the master's thesis proposal has to be prepared and must be accepted by the advisor(s). The master's thesis proposal must follow the respective template of the dean of academic affairs. Thereafter the proposal has to be applied for at the deanery. The necessary forms may be found on the web site of the Faculty of Informatics. \url{http://www.informatik.tuwien.ac.at/dekanat/formulare.html}
  \item  Accompanied with the master's thesis proposal, the structure of the thesis in terms of a table of contents has to be provided.
  \item Then, the first talk has to be given at the so-called ``Seminar for Master Students''. The slides have to be discussed with the advisor(s) one week in advance. Attendance of the ``Seminar for Master Students'' is compulsory and offers the opportunity to discuss arising problems among other master students.
  \item At the latest five months after the beginning, a provisional final version of the thesis has to be handed over to the advisor(s).  
  \item As soon as the provisional final version exists, a first poster draft has to be made. The making of a poster is a compulsory part of the ``Seminar for Master Students'' for all master studies at the Faculty of Informatics. Drafts and design guidelines can be found at \url{http://www.informatik.tuwien.ac.at/studium/richtlinien}.
  \item After having consulted the advisor(s) the second talk has to be held at the ``Seminar for Master Students''.
  \item At the latest six months after the beginning, the corrected version of the master's thesis and the poster have to be handed over to the advisor(s).
  \item After completion the master's thesis has to be presented at the ``epilog''. For detailed information on the epilog see: \\ \url{http://www.informatik.tuwien.ac.at/studium/epilog}
\end{enumerate}

\section{Structure of the Master's Thesis}

If the curriculum regulates the language of the master's thesis to be English (like for ``Business Informatics''), the thesis has to be written in English. Otherwise, the master's thesis may be written in English or in German. The structure of the thesis is predetermined.
The table of contents is followed by the introduction and the main part, which can vary according to the content. The master's thesis ends with the bibliography (compulsory) and the appendix (optional).

\begin{itemize}
  \item	Cover page
  \item Acknowledgements
  \item Abstract of the thesis in English and German
  \item Table of contents
  \item Introduction
  	\begin{itemize}
  		\item motivation
  		\item problem statement (which problem should be solved?)
  		\item aim of the work
  		\item methodological approach
  		\item structure of the work
  	\end{itemize}
  \item State of the art / analysis of existing approaches
  	\begin{itemize}
  		\item literature studies
  		\item analysis
  		\item comparison and summary of existing approaches
  	\end{itemize}
  \item Methodology
  	\begin{itemize}
  		\item used concepts
  		\item methods and/or models
  		\item languages
  		\item design methods
  		\item data models
  		\item analysis methods
  		\item formalisms
  	\end{itemize}
  \item Suggested solution/implementation
  \item Critical reflection
  	\begin{itemize}
  		\item comparison with related work
  		\item discussion of open issues
  	\end{itemize}
  \item Summary and future work
  \item Appendix: source code, data models, \dots
  \item Bibliography
\end{itemize}

